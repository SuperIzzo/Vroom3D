\documentclass[11pt,fleqn,twoside]{article}
\usepackage{makeidx}
\makeindex
\usepackage{palatino} %or {times} etc
\usepackage{plain} %bibliography style 
\usepackage{amsmath} %math fonts - just in case
\usepackage{amsfonts} %math fonts
\usepackage{amssymb} %math fonts
\usepackage{lastpage} %for footer page numbers
\usepackage{fancyhdr} %header and footer package
\usepackage{mmpv2} 
\usepackage{url}

% the following packages are used for citations - You only need to include one. 
%
% Use the cite package if you are using the numeric style (e.g. IEEEannot). 
% Use the natbib package if you are using the author-date style (e.g. authordate2annot). 
% Only use one of these and comment out the other one. 
\usepackage{cite}
%\usepackage{natbib}

\begin{document}

\name{Hristoz Stefnov Stefanov}
\userid{hzs9}
\projecttitle{3D Volume Rendering Engine for Interactive Applications}
\projecttitlememoir{3D Volume Rendering Engine for Interactive Applications} %same as the project title or abridged version for page header
\reporttitle{Progress Report}
\version{0.1}
\docstatus{Draft}
\modulecode{CS39440}
\supervisor{Bernie Tiddeman} % e.g. Neil Taylor
\supervisorid{bpt}
\degreeschemecode{G451}
\degreeschemename{Computer Graphics, Vision And Games (Inc Integrated Industrial And Professional Training)}
\wordcount{4500}

%optional - comment out next line to use current date for the document
\documentdate{18th November 2012} 
\mmp

\setcounter{tocdepth}{3} %set required number of level in table of contents
\tableofcontents

\newpage

%==============================================================================
\section{Project Summary}
%==============================================================================
Text in here.

%==============================================================================
\section{Current Progress}
%==============================================================================


\subsection{Research}
The predominant part of my work so far has been research. Initially I was concerned about the relevancy of my project as volume rendering is not new technology. I started researching the already existing solutions placing more importance on open-source ones, in the event of finding a suitable one I would use it as a starting point for my project. Indeed I did find a couple of good engines, which however were not suitable for one reason or another.

Voreen - GPL, targeted at researchers, non-commercial use which contradicts with my goal - open source volume rendering engine which encourages people to use it for whatever purpose they need
Voxlap - source incredibly cryptic, no comments, magic numbers and syntax I've never seen
Voxrend - minimalistic ray-tracer, written in C, sourcecode in spanish
PolyVox - minecraft like geared voxel engine - i.e. cubes


UnitTest++
Eigen

%==============================================================================
\section{Planning}
%==============================================================================
Text in here.



\nocite{*} % include everything from the bibliography, irrespective of whether it has been referenced.

% the following line is included so that the bibliography is also shown in the table of contents. There is the possibility that this is added to the previous page for the bibliography. To address this, a newline is added so that it appears on the first page for the bibliography. 
\newpage
\addcontentsline{toc}{section}{Annotated Bibliography} 

%
% example of including an annotated bibliography. The current style is an author date one. If you want to change, comment out the line and uncomment the subsequent line. You should also modify the packages included at the top (see the notes earlier in the file) and then trash your aux files and re-run. 
%\bibliographystyle{authordate2annot}
\bibliographystyle{IEEEannot}
\renewcommand{\refname}{Annotated Bibliography}  % if you put text into the final {} on this line, you will get an extra title, e.g. References. This isn't necessary for the outline project specification. 
\bibliography{mmp} % References file


\end{document}